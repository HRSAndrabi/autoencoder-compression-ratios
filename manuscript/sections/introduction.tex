% Introduction


\noindent
Ours is the age of information: where digital data is ubiquitous, and computational resources are in consequently constrained supply.
In this climate, a fundamental problem persists in the efficient management of data to limit the computational expense of information storage and exchange.
% Indeed, the speed at which information can be exchanged is constrained by the limits of our computational systems, and the ingenuity of data-management techniques to reduce the computational expense of information exchange.
On this account, data-compression techniques seek to alleviate the computational burden of sending and storing digital information by compressing data to representations of reduced size.
In particular, practical applications of such techniques can be broadly divided as employing either lossless, or lossy compression.
Lossless techniques comprise the more common class of compression algorithms, which target perfect reconstruction of input data from compressed formats.
While such techniques offer uncompromised data-accuracy between compressed and uncompressed formats, recent reports show that practical applications rarely achieve compression ratios in excess of two- to four-times the size of the original data \cite{mittal2015survey}.
On the other hand, lossy data-compression techniques are able to achieve exceptionally high compression ratios at the expense of degraded data accuracy.
High compression-ratios substantially lower file sizes, and thereby reduce the computational expense of storing and sending information. 
Accordingly, in scenarios where speed of information exchange is paramount and minor degradation in data-accuracy is unlikely to be noticed by end-users (such as in the case of image-based exchanges), lossy compression becomes an attractive approach. 

To this end, among the most successful non-probabalistic approaches to lossy compression is through the use of autoencoder networks.
Autoencoders are representation-learning applications of unsupervised artificial neural networks (ANNs), that seek to learn mappings of high-dimensional data to meaningful lower-dimensional representations.
By learning fundamental representations of data, autoencoders enable fast transformation between highly compressed and uncompressed formats, and thereby enable storage and exchange of large files in compressed and computationally inexpensive formats.
Indeed, several works demonstrate the ability of autoencoders to learn compressed representations of image data, with promising reconstruction accuracy \cite{cheng2018deep, toderici2017full, balle2016end}.
Despite the interest autoencoder networks have generated, little  documentation of the effect of variation in model hyperparameters has emerged to this date.
Recognising this, the analysis in this note seeks to evaluate the reconstruction accuracy of various autoencoder network architectures to image compression of high-dimensional image data. 
To this end, the present analysis will investigate variations in reconstruction-error resulting from modifications to compression-ratios enforced by the autoencoder, and inclusion or exclusion of bias nodes in the network. In particular, the analysis will address the following research questions:

\begin{itemize}
	\item[] \textbf{RQ1}: How does the selection of compression ratio influence reconstruction accuracy of compressed images?
	\item[] \textbf{RQ2}: How does the inclusion/exclusion of bias nodes affect reconstruction accuracy of compressed images?
\end{itemize}

\noindent
The remainder of the paper is structures as follows. Section 2 provides an overview of the dataset used in this analysis. Section 3 describes the general experimental framework, and the employed model estimation methodology. Section 4 presents and discusses the results. Finally, Section 5 concludes the paper.



